%%%%%%%%%%%%%%%%%%%%%%%%%%%%%%%%%%%%%%%%%
% Simons CV based on '"ModernCV" CV and Cover Letter'
% LaTeX Template
% Version 1.11 (19/6/14)
%
% Simons CV. I slimmed down moderncv to my needs, missusing it most likely
% heavily, but still does it's job nicely :-)
%
% author:
% Simon Schiele (simon.codingmonkey@gmail.com)
%
% The original template has been downloaded from:
% http://www.LaTeXTemplates.com
%
% Original author:
% Xavier Danaux (xdanaux@gmail.com)
%
% License:
% CC BY-NC-SA 3.0 (http://creativecommons.org/licenses/by-nc-sa/3.0/)
%
% Important note:
% This template requires the moderncv.cls and .sty files to be in the same
% directory as this .tex file. These files provide the resume style and themes
% used for structuring the document.
%
%%%%%%%%%%%%%%%%%%%%%%%%%%%%%%%%%%%%%%%%%

%----------------------------------------------------------------------------------------
%	PACKAGES AND OTHER DOCUMENT CONFIGURATIONS
%----------------------------------------------------------------------------------------

\documentclass[11pt,a4paper,sans]{moderncv} % Font sizes: 10, 11, or 12; paper sizes: a4paper, letterpaper, a5paper, legalpaper, executivepaper or landscape; font families: sans or roman

\moderncvstyle{classic} % CV theme - options include: 'casual' (default), 'classic', 'oldstyle' and 'banking'
\moderncvcolor{blue} % CV color - options include: 'blue' (default), 'orange', 'green', 'red', 'purple', 'grey' and 'black'

\usepackage{lipsum} % Used for inserting dummy 'Lorem ipsum' text into the template
\usepackage{ngerman}


\usepackage[scale=0.75]{geometry} % Reduce document margins
%\setlength{\hintscolumnwidth}{3cm} % Uncomment to change the width of the dates column
%\setlength{\makecvtitlenamewidth}{10cm} % For the 'classic' style, uncomment to adjust the width of the space allocated to your name

%----------------------------------------------------------------------------------------
%	NAME AND CONTACT INFORMATION SECTION
%----------------------------------------------------------------------------------------

\firstname{Simon} % Your first name
\familyname{Schiele} % Your last name

% All information in this block is optional, comment out any lines you don't need
\title{Curriculum Vitae}
% \title{Lebenslauf}
\address{Vilbeler Landstra"se 83}{60388 Frankfurt}
\mobile{(+49) 152 5330 7817}
% \phone{(000) 111 1112}
% \fax{(000) 111 1113}
\email{simon.codingmonkey@gmail.com}
% \homepage{http://simon.psaux.de}{http://simon.psaux.de/} % The first argument is the url for the clickable link, the second argument is the url displayed in the template
% \extrainfo{additional information}
\photo[70pt][0.4pt]{pictures/picture} % The first bracket is the picture height, the second is the thickness of the frame around the picture (0pt for no frame)
% \quote{``Never allow the same bug to bite you twice.''- Steve Maguire}

%----------------------------------------------------------------------------------------

\begin{document}

\setlength{\hintscolumnwidth}{0.19\textwidth}
\setlength{\parskip}{20em}

\makecvtitle % Print the CV title

\section{Technische Kenntnisse\newline{}}

\cvitemwithcomment{Programmierung}{Python, PHP, Bash}{(Level: sehr gut)}
\cvitemwithcomment{}{JavaScript}{(Level: gut)}
\cvitemwithcomment{Datenbanken}{PostgreSQL, MySQL, SQLite, MongoDB}{}
\cvitemwithcomment{Entwicklungstools}{Git, Subversion, Mercurial, Jira, Confluence}{}
\cvitemwithcomment{Testing}{Jenkins, Git-Hooks, pytest, python-unittest}{}
\cvitemwithcomment{}{Selenium, PhantomJS, Apache ab\newline{}\newline{}}{}

%----------------------------------------------------------------------------------------
%	WORK EXPERIENCE SECTION
%----------------------------------------------------------------------------------------

\section{Beruflicher Werdegang\newline{}}

\cventry{06/2009--02/2015}{Software Developer}{\textsc{Ypsilon.Net AG}}{Frankfurt}{}{
\newline{}\newline{}
\begin{itemize}
\item Entwicklung von Backend-Anwendungen und Webservices im Tourismus-\newline{}und Finanzbereich (Python, PHP)
\item Aufbau und Pflege eines PCI DSS Scope (Class 1)
\item Administration von Application-, Entwicklungs- und Webservern
\newline{}\newline{}
\end{itemize}}

\cventry{06/2006--12/2008}{Software Engineer}{\textsc{ReflexMedia GmbH}}{Limburg}{}{
\newline{}\newline{}
\begin{itemize}
\item Entwicklung eines POS Multimedia Systems auf Linux Basis
\item Aufbau und Pflege der firmeneigenen GNU/Linux Distribution (Debian basiert)
\item Administration und Pflege mehrerer hundert Client Systeme\newline{}(Dezentral, Deutschlandweit)
\item Administration der Linux Web-, Entwicklungs- und Packetserver
\newline{}\newline{}
\end{itemize}}

\cventry{06/2005--12/2005}{Mediengestalter}{\textsc{ReflexMedia GmbH}}{Limburg}{}{6 Monatiges Praktikum im Rahmen der Ausbildung zum Mediengestalter
% \newline{}\newline{}
\begin{itemize}
\item Entwicklung einer Demonstartionsumgebung f"ur dynamische POS Werbung (Flash/ActionScript, PHP)
\item Weiterentwicklung von bestehenden Webshops (Bash, PHP)
\newline{}\newline{}
\end{itemize}}

\cventry{06/2001--06/2002}{Data Center Administrator}{\textsc{Innovative Software AG}}{Frankfurt}{}{Jahrespraktikum im Rahmen der Fachoberschule
% \newline{}\newline{}
\begin{itemize}
\item Administration von etwa 1000 Produktivservern (RedHat GNU/Linux)
\item Vor-Ort Service in diversen Rechenzentren im Frankfurter Raum
\item Pflege und Erweiterung des Administratoren Intranets (PHP, JavaScript)
\newline{}\newline{}
\end{itemize}}

%----------------------------------------------------------------------------------------
%	EDUCATION SECTION
%----------------------------------------------------------------------------------------

\section{Ausbildung\newline{}}

\cventry{2004--2006}{Ausbildung: Mediengestaltung Digital- und Printmedien}{IAD}{Marburg}{}{Abschluss: Mediengestalter (IHK)\newline{}}  % Arguments not required can be left empty
\cventry{2001--2004}{Fachoberschule}{}{B"udingen}{}{Fachrichtung: Wirtschaftsinformatik\newline{}}  % Arguments not required can be left empty
\cventry{1995--2001}{Realschule}{Schule am Dohlberg}{B"udingen}{}{Abschluss: Mittlere Reife\newline{}}

%----------------------------------------------------------------------------------------
%	LANGUAGES SECTION
%----------------------------------------------------------------------------------------

\section{Sprachen\newline{}}

\cvitemwithcomment{Deutsch}{Muttersprache}{}
\cvitemwithcomment{Englisch}{Fl"ussig in Wort und Schrift\newline{}}{}

%----------------------------------------------------------------------------------------
%	INTERESTS SECTION
%----------------------------------------------------------------------------------------

\section{Pers"onliche Interessen\newline{}}

\renewcommand{\listitemsymbol}{-~} % Changes the symbol used for lists

\cvlistdoubleitem{Kochen}{Britische Fernsehserien}
% Python PHP Bash Lua Git SCM Subversion PostgreSQL MySQL MongoDB JavaScript jQuery AJAX Jenkins OpenSSL Online Payment PCI-DSS MVC HTML XML CSS VimL DevOps git-flow

\end{document}
